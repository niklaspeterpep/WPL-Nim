\documentclass{beamer}
\usetheme{Berlin}
\title{Nim}
\subtitle{Workshop: Programming Languages}
\author{Niklas Peter}
\institute{Universität Siegen}
\date{5. Februar 2024}
    
\begin{document}


\begin{frame}
\titlepage
\end{frame}


\begin{frame}
\frametitle{Inhalt}
\tableofcontents
\end{frame}


\begin{frame}
\frametitle{Allgemeines}
\section{Allgemeines}
\begin{itemize}
	\item Andreas Rumpf startete 2005 die Entwicklung
	\item Veröffentlichung in 2008 unter MIT Lizenz
	\item Version 1.0 kam am 23.09.2019 raus
	\item Soll effizient, ausdrucksvoll und elegant sein
	\item Generiert native ausführbare Programme ohne Abhängigkeiten
\end{itemize}
\end{frame}


\begin{frame}
\frametitle{Design}
\section{Design}
\begin{itemize}
	\item Syntax inspiriert von Python (Einrückung, Schlüsselwörter)
	\item Style-Insensitive für Mischung verschiedener Stile
	
	\emph{ein\_beispiel} und \emph{einBeispiel} sind gleich
	\item Statische Typisierung aber mit einfacher Typumwandlung
	\item Uniform Function Call Syntax (UFCS)
	
	\emph{x = multiply(y, z)} entspricht \emph{x = y.multiply(z)}
	\item Programmierungsparadigmen
	\begin{itemize}
		\item Funktionale Programmierung
		\item Objektorientierte Programmierung
		\item Metaprogrammierung
		\item Foreign Function Interface (FFI)
		\item Parallelisierung und Nebenläufigkeit
	\end{itemize}
\end{itemize}
\end{frame}


\begin{frame}
\frametitle{Infrastruktur}
\section{Infrastruktur}
\begin{itemize}
	\item Kompilierer
	\begin{itemize}
		\item Vollständig in Nim geschrieben
		\item Gibt optimierten C Code aus. Kompilierung zu C++, Objective-C und JavaScript ebenfalls möglich
		\item Objektcode durch externen Kompilierer wie GCC
		\item Verschiedene Garbage Collectors verfügbar
	\end{itemize}
	\item Entwicklungswerkzeuge (Auswahl)
	\begin{itemize}
		\item Nim's Paketmanager Nimble
		\item Nimsuggest zur Integration in IDEs
		\item Hot code reloading zum Neuladen des Codes zur Laufzeit
	\end{itemize}
	\item Bibliotheken
	\begin{itemize}
		\item Pure und Impure Bibliotheken
		\item Standardbibliothek mit grundlegenden Funktionen vollständig in Nim
		\item Einbindung von Bibliotheken in C, C++, Objective-C und JavaScript möglich
	\end{itemize}
\end{itemize}
\end{frame}


\begin{frame}
\frametitle{Anwendungsfälle}
\section{Anwendungsfälle}
Viele Einsatzmöglichkeiten durch Kompilierung nach C, C++ und JavaScript:
\begin{itemize}
	\item Shell Scripting
	\item Front- und Backend für Webdienste
	\item Machine Learning
	\item Eingebettete Systeme
\end{itemize}

Prominentes Beispiel ist \textbf{nitter}, ein Frontend für X (Twitter)
\end{frame}


\begin{frame}
\frametitle{Stärken und Schwächen}
\section{Stärken und Schwächen}
\begin{itemize}
	\item Einfache und flexible Syntax
	\item Moderne Konzepte
	\item Viele Anwendungsbereiche
	\item Nicht für alles eine native Bibliothek vorhanden, dafür aber Import von Bibliotheken in anderen Sprachen möglich
	\item Gute Dokumentation
\end{itemize}
\end{frame}


\begin{frame}
\frametitle{Implementierung}
\section{Implementierung}
\begin{itemize}
	\item Punkt
\end{itemize}
\end{frame}

\appendix

\begin{frame}
\frametitle{Vielen Dank für Ihre Aufmerksamkeit!}
Fragen?
\end{frame}


\end{document}


